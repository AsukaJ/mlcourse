%% LyX 2.3.2 created this file.  For more info, see http://www.lyx.org/.
%% Do not edit unless you really know what you are doing.
\documentclass[english,aspectratio=169]{beamer}
\usepackage{mathptmx}
\usepackage{eulervm}
\usepackage[T1]{fontenc}
\usepackage[latin9]{inputenc}
\usepackage{babel}
\usepackage{amsmath}
\usepackage{amssymb}
\usepackage{dsfont}
\ifx\hypersetup\undefined
  \AtBeginDocument{%
    \hypersetup{unicode=true,pdfusetitle,
 bookmarks=true,bookmarksnumbered=false,bookmarksopen=false,
 breaklinks=false,pdfborder={0 0 0},pdfborderstyle={},backref=false,colorlinks=true,
 allcolors=NYUPurple,urlcolor=LightPurple}
  }
\else
  \hypersetup{unicode=true,pdfusetitle,
 bookmarks=true,bookmarksnumbered=false,bookmarksopen=false,
 breaklinks=false,pdfborder={0 0 0},pdfborderstyle={},backref=false,colorlinks=true,
 allcolors=NYUPurple,urlcolor=LightPurple}
\fi

\makeatletter

%%%%%%%%%%%%%%%%%%%%%%%%%%%%%% LyX specific LaTeX commands.
%% Because html converters don't know tabularnewline
\providecommand{\tabularnewline}{\\}

%%%%%%%%%%%%%%%%%%%%%%%%%%%%%% Textclass specific LaTeX commands.
% this default might be overridden by plain title style
\newcommand\makebeamertitle{\frame{\maketitle}}%
% (ERT) argument for the TOC
\AtBeginDocument{%
  \let\origtableofcontents=\tableofcontents
  \def\tableofcontents{\@ifnextchar[{\origtableofcontents}{\gobbletableofcontents}}
  \def\gobbletableofcontents#1{\origtableofcontents}
}

%%%%%%%%%%%%%%%%%%%%%%%%%%%%%% User specified LaTeX commands.
\usetheme{CambridgeUS} 
\beamertemplatenavigationsymbolsempty


% Set Color ==============================
\definecolor{NYUPurple}{RGB}{87,6,140}
\definecolor{LightPurple}{RGB}{165,11,255}


\setbeamercolor{title}{fg=NYUPurple}
%\setbeamercolor{frametitle}{fg=NYUPurple}
\setbeamercolor{frametitle}{fg=NYUPurple}

\setbeamercolor{background canvas}{fg=NYUPurple, bg=white}
\setbeamercolor{background}{fg=black, bg=NYUPurple}

\setbeamercolor{palette primary}{fg=black, bg=gray!30!white}
\setbeamercolor{palette secondary}{fg=black, bg=gray!20!white}
\setbeamercolor{palette tertiary}{fg=gray!20!white, bg=NYUPurple}

\setbeamertemplate{headline}{}

\setbeamercolor{parttitle}{fg=NYUPurple}
\setbeamercolor{sectiontitle}{fg=NYUPurple}
\setbeamercolor{sectionname}{fg=NYUPurple}
\setbeamercolor{section page}{fg=NYUPurple}

\AtBeginSection[]{
  \begin{frame}
  \vfill
  \centering
\setbeamercolor{section title}{fg=NYUPurple}
 \begin{beamercolorbox}[sep=8pt,center,shadow=true,rounded=true]{title}
    \usebeamerfont{title}\usebeamercolor[fg]{title}\insertsectionhead\par%
  \end{beamercolorbox}
  \vfill
  \end{frame}
}

\makeatother

\begin{document}
\global\long\def\reals{\mathbf{R}}%
 
\global\long\def\integers{\mathbf{Z}}%
 
\global\long\def\naturals{\mathbf{N}}%
 
\global\long\def\rationals{\mathbf{Q}}%
 
\global\long\def\ca{\mathcal{A}}%
 
\global\long\def\cb{\mathcal{B}}%
 
\global\long\def\cc{\mathcal{C}}%
 
\global\long\def\cd{\mathcal{D}}%
 
\global\long\def\ce{\mathcal{E}}%
 
\global\long\def\cf{\mathcal{F}}%
 
\global\long\def\cg{\mathcal{G}}%
 
\global\long\def\ch{\mathcal{H}}%
 
\global\long\def\ci{\mathcal{I}}%
 
\global\long\def\cj{\mathcal{J}}%
 
\global\long\def\ck{\mathcal{K}}%
 
\global\long\def\cl{\mathcal{L}}%
 
\global\long\def\cm{\mathcal{M}}%
 
\global\long\def\cn{\mathcal{N}}%
 
\global\long\def\co{\mathcal{O}}%
 
\global\long\def\cp{\mathcal{P}}%
 
\global\long\def\cq{\mathcal{Q}}%
 
\global\long\def\calr{\mathcal{R}}%
 
\global\long\def\cs{\mathcal{S}}%
 
\global\long\def\ct{\mathcal{T}}%
 
\global\long\def\cu{\mathcal{U}}%
 
\global\long\def\cv{\mathcal{V}}%
 
\global\long\def\cw{\mathcal{W}}%
 
\global\long\def\cx{\mathcal{X}}%
 
\global\long\def\cy{\mathcal{Y}}%
 
\global\long\def\cz{\mathcal{Z}}%
 
\global\long\def\ind#1{1(#1)}%
 %\newcommand{\pr}{P}
\global\long\def\pr{\mathbb{P}}%
 
\global\long\def\predsp{\cy}%
 %{\hat{\cy}}
\global\long\def\outsp{\cy}%

\global\long\def\prxy{P_{\cx\times\cy}}%
 
\global\long\def\prx{P_{\cx}}%
 
\global\long\def\prygivenx{P_{\cy\mid\cx}}%
 %\newcommand{\ex}{E}
\global\long\def\ex{\mathbb{E}}%
 
\global\long\def\var{\textrm{Var}}%
 
\global\long\def\cov{\textrm{Cov}}%
 
\global\long\def\sgn{\textrm{sgn}}%
 
\global\long\def\sign{\textrm{sign}}%
 
\global\long\def\kl{\textrm{KL}}%
 
\global\long\def\law{\mathcal{L}}%
 
\global\long\def\eps{\varepsilon}%
 
\global\long\def\as{\textrm{ a.s.}}%
 
\global\long\def\io{\textrm{ i.o.}}%
 
\global\long\def\ev{\textrm{ ev.}}%
 
\global\long\def\convd{\stackrel{d}{\to}}%
 
\global\long\def\eqd{\stackrel{d}{=}}%
 
\global\long\def\del{\nabla}%
 
\global\long\def\loss{\ell}%
 
\global\long\def\risk{R}%
 
\global\long\def\emprisk{\hat{R}}%
 
\global\long\def\lossfnl{L}%
 
\global\long\def\emplossfnl{\hat{L}}%
 
\global\long\def\empminimizer#1{\hat{#1}^{*}}%
 
\global\long\def\minimizer#1{#1^{*}}%
\global\long\def\optimizer#1{#1^{*}}%
 
\global\long\def\etal{\textrm{et. al.}}%
 
\global\long\def\tr{\operatorname{tr}}%

\global\long\def\trace{\operatorname{trace}}%
 
\global\long\def\diag{\text{diag}}%
 
\global\long\def\rank{\text{rank}}%
 
\global\long\def\linspan{\text{span}}%
 
\global\long\def\spn{\text{span}}%
 
\global\long\def\proj{\text{Proj}}%
 
\global\long\def\argmax{\operatornamewithlimits{arg\, max}}%
 
\global\long\def\argmin{\operatornamewithlimits{arg\, min}}%

\global\long\def\bfx{\mathbf{x}}%
 
\global\long\def\bfy{\mathbf{y}}%
 
\global\long\def\bfl{\mathbf{\lambda}}%
 
\global\long\def\bfm{\mathbf{\mu}}%
 
\global\long\def\calL{\mathcal{L}}%

\global\long\def\vw{\boldsymbol{w}}%
 
\global\long\def\vx{\boldsymbol{x}}%
 
\global\long\def\vxi{\boldsymbol{\xi}}%
 
\global\long\def\valpha{\boldsymbol{\alpha}}%
 
\global\long\def\vbeta{\boldsymbol{\beta}}%
 
\global\long\def\vsigma{\boldsymbol{\sigma}}%
\global\long\def\vtheta{\boldsymbol{\theta}}%
 
\global\long\def\vd{\boldsymbol{d}}%
 
\global\long\def\vs{\boldsymbol{s}}%
 
\global\long\def\vt{\boldsymbol{t}}%
 
\global\long\def\vh{\boldsymbol{h}}%
 
\global\long\def\ve{\boldsymbol{e}}%
 
\global\long\def\vf{\boldsymbol{f}}%
 
\global\long\def\vg{\boldsymbol{g}}%
 
\global\long\def\vz{\boldsymbol{z}}%
 
\global\long\def\vk{\boldsymbol{k}}%
 
\global\long\def\va{\boldsymbol{a}}%
 
\global\long\def\vb{\boldsymbol{b}}%
 
\global\long\def\vv{\boldsymbol{v}}%
 
\global\long\def\vy{\boldsymbol{y}}%

\global\long\def\dom{\textrm{\textbf{dom} }}%
\global\long\def\rank{\text{\textbf{rank }}}%
\global\long\def\conv{\textrm{\textbf{conv} }}%
\global\long\def\relint{\text{\textbf{relint }}}%
\global\long\def\aff{\text{\textbf{aff }}}%

\global\long\def\hil{\ch}%
 
\global\long\def\rkhs{\hil}%
 
\global\long\def\ber{\text{Ber}}%

\title[DS-GA 1003 / CSCI-GA 2567]{Introduction to Structured Prediction and Recap of Bayesian}
\author{Xintian Han}
\date{April 3, 2019}
\institute{CDS, NYU}

\makebeamertitle
\mode<article>{Just in article version}

\begin{frame}{Contents}

\tableofcontents{}
\end{frame}


\section{Introduction to Structured Prediction}
\begin{frame}{Multiclass Hypothesis Space: Reframed}

\begin{itemize}
\item \textbf{General {[}Discrete{]} Output Space: }$\cy$ 

\pause{}
\item \textbf{Base Hypothesis Space: $\ch=\left\{ h:\cx\times\cy\to\reals\right\} $}
\begin{itemize}
\item $h(x,y)$ gives \textbf{compatibility score} between input $x$ and
output $y$ 
\end{itemize}

\pause{}
\item \textbf{Multiclass Hypothesis Space}
\[
\cf=\left\{ x\mapsto\argmax_{y\in\cy}h(x,y)\mid h\in\ch\right\} 
\]


\pause{}
\item Final prediction function is an $f\in\cf$. 
\item For each $f\in\cf$ there is an underlying compatibility score function
$h\in\ch$. 
\end{itemize}
\end{frame}
%

\begin{frame}{Part-of-speech (POS) Tagging}

\begin{itemize}
\item Given a sentence, give a part of speech tag for each word:
\end{itemize}
\begin{center}
\begin{tabular}{|c|c|c|c|c|}
\hline 
$x$ & $\underbrace{\mbox{[START]}}_{x_{0}}$ & $\underbrace{\mbox{He}}_{x_{1}}$ & $\underbrace{\mbox{eats}}_{x_{2}}$ & $\underbrace{\mbox{apples}}_{x_{3}}$\tabularnewline
\hline 
$y$ & $\underbrace{\mbox{[START]}}_{y_{0}}$ & $\underbrace{\mbox{Pronoun}}_{y_{1}}$ & $\underbrace{\mbox{Verb}}_{y_{2}}$ & $\underbrace{\mbox{Noun}}_{y_{3}}$\tabularnewline
\hline 
\end{tabular}
\par\end{center}

\pause{}
\begin{itemize}
\item $\cv=\left\{ \mbox{all English words}\right\} \cup\left\{ \mbox{[START]},"."\right\} $

\pause{}
\item $\cp=\left\{ \mbox{START},\mbox{Pronoun,Verb,Noun,Adjective}\right\} $

\pause{}
\item $\cx=\cv^{n}$, $n=1,2,3,\ldots$ {[}Word sequences of any length{]}

\pause{}
\item $\cy=\cp^{n},\,n=1,2,3,\ldots${[}Part of speech sequence of any length{]}
\end{itemize}
\end{frame}
%

%
\begin{frame}{Structured Prediction}

\begin{itemize}
\item A \textbf{structured prediction }problem is a multiclass problem in
which $\cy$ is very large, but has (or we assume it has) a certain
structure.
\end{itemize}

\pause{}
\begin{itemize}
\item For POS tagging, $\cy$ grows exponentially in the length of the sentence.
\end{itemize}

\pause{}
\begin{itemize}
\item Typical \textbf{structure} assumption: The POS labels form a Markov
chain.
\begin{itemize}
\item i.e. $y_{n+1}\mid y_{n},y_{n-1},\ldots,y_{0}$ is the same as $y_{n+1}\mid y_{n}$.
\end{itemize}
\end{itemize}
\end{frame}
%
\begin{frame}{Local Feature Functions: Type 1}

\begin{itemize}
\item A ``type 1'' \textbf{local feature }only depends on 
\begin{itemize}
\item the label at a single position, say $y_{i}$ (label of the $i$th
word) and
\item $x$ at any position
\end{itemize}
\end{itemize}

\pause{}
\begin{itemize}
\item Example: 
\begin{eqnarray*}
\phi_{1}(i,x,y_{i}) & = & \ind{x_{i}=\mbox{runs}}\ind{y_{i}=\mbox{Verb}}\\
\pause\phi_{2}(i,x,y_{i}) & = & \ind{x_{i}=\mbox{runs}}\ind{y_{i}=\mbox{Noun}}\\
\pause\phi_{3}(i,x,y_{i}) & = & \ind{x_{i-1}=\mbox{He}}\ind{x_{i}=\mbox{runs}}\ind{y_{i}=\mbox{Verb}}
\end{eqnarray*}
 
\end{itemize}
\end{frame}
%
\begin{frame}{Local Feature Functions: Type 2}

\begin{itemize}
\item A ``type 2'' \textbf{local feature }only depends on 
\begin{itemize}
\item the labels at 2 consecutive positions: $y_{i-1}$ and $y_{i}$
\item $x$ at any position
\end{itemize}
\end{itemize}

\pause{}
\begin{itemize}
\item Example: 
\begin{eqnarray*}
\theta_{1}(i,x,y_{i-1},y_{i}) & = & \ind{y_{i-1}=\mbox{Pronoun}}\ind{y_{i}=\mbox{Verb}}\\
\theta_{2}(i,x,y_{i-1},y_{i}) & = & \ind{y_{i-1}=\mbox{Pronoun}}\ind{y_{i}=\mbox{Noun}}
\end{eqnarray*}
 
\end{itemize}
\end{frame}
%
\begin{frame}{Local Feature Vector and Compatibility Score}

\begin{itemize}
\item At each position $i$ in sequence, define the \textbf{local feature
vector}:
\begin{eqnarray*}
\Psi_{i}(x,y_{i-1},y_{i}) & = & (\phi_{1}(i,x,y_{i}),\phi_{2}(i,x,y_{i}),\ldots,\\
 &  & \theta_{1}(i,x,y_{i-1},y_{i}),\theta_{2}(i,x,y_{i-1},y_{i}),\ldots)
\end{eqnarray*}
\end{itemize}

\pause{}
\begin{itemize}
\item \textbf{Local compatibility score} for $(x,y)$ at position $i$ is
$\left\langle w,\Psi_{i}(x,y_{i-1},y_{i})\right\rangle $. 
\end{itemize}
\end{frame}
%
\begin{frame}{Sequence Compatibility Score}

\begin{itemize}
\item The \textbf{compatibility score} for the pair of sequences $\left(x,y\right)$
is the sum of the local compatibility scores: 
\begin{eqnarray*}
 &  & \sum_{i}\left\langle w,\Psi_{i}(x,y_{i-1},y_{i})\right\rangle \\
\pause & = & \left\langle w,\sum_{i}\Psi_{i}(x,y_{i-1},y_{i})\right\rangle \\
\pause & = & \left\langle w,\Psi(x,y)\right\rangle ,
\end{eqnarray*}
where we define the sequence feature vector by 
\[
\Psi(x,y)=\sum_{i}\Psi_{i}(x,y_{i-1},y_{i}).
\]
\end{itemize}

\pause{}
\begin{itemize}
\item So we see this is a special case of linear multiclass prediction.
\end{itemize}
\end{frame}
%
\begin{frame}{Sequence Target Loss}

\begin{itemize}
\item How do we assess the loss for prediction sequence $y'$ for example
$\left(x,y\right)?$ 
\end{itemize}

\pause{}
\begin{itemize}
\item \textbf{Hamming loss} is common:
\[
\Delta(y,y')=\frac{1}{\left|y\right|}\sum_{i=1}^{\left|y\right|}\ind{y_{i}\neq y_{i}'}
\]
\end{itemize}

\pause{}
\begin{itemize}
\item Could generalize this as
\[
\Delta(y,y')=\frac{1}{\left|y\right|}\sum_{i=1}^{\left|y\right|}\delta(y_{i},y_{i}')
\]
 
\end{itemize}
\end{frame}
%
\begin{frame}{What remains to be done?}

\begin{itemize}
\item To compute predictions, we need to find 
\[
\argmax_{y\in\cy}\left\langle w,\Psi(x,y)\right\rangle .
\]
\item This is straightforward for $|\cy|$ small. 
\end{itemize}

\pause{}
\begin{itemize}
\item Now $\left|\cy\right|$ is exponentially large.
\end{itemize}

\pause{}
\begin{itemize}
\item Because $\Psi$ breaks down into local functions only depending on
$2$ adjacent labels,
\begin{itemize}
\item we can solve this efficiently using dynamic programming. 
\item (Similar to Viterbi decoding.)
\end{itemize}

\pause{}
\item Learning can be done with SGD and a similar dynamic program.
\end{itemize}
\end{frame}
\section{Recap: Bayesian Methods}
\begin{frame}{Bayesian Decision Theory}
\begin{itemize}
\item Ingredients: 
\begin{itemize}
\item \textbf{Parameter space} $\Theta$.
\item \textbf{Prior}: Distribution $p(\theta)$ on $\Theta$.
\item \textbf{Action space} $\ca$.
\item \textbf{Loss function}: $\ell:\ca\times\Theta\to\reals$.

\pause{}
\end{itemize}
\item The \textbf{posterior risk} of an action $a\in\ca$ is 
\begin{eqnarray*}
r(a) & := & \ex\left[\ell(\theta,a)\mid\cd\right]\\
\pause & = & \int\ell(\theta,a)p(\theta\mid\cd)\,d\theta.
\end{eqnarray*}


\pause{}
\begin{itemize}
\item It's the \textbf{expected loss under the posterior.}

\pause{}
\end{itemize}
\item A \textbf{Bayes action} $a^{*}$ is an action that minimizes posterior
risk:
\[
r(a^{*})=\min_{a\in\ca}r(a)
\]
\end{itemize}
\end{frame}
%
\begin{frame}{The Posterior Predictive Distribution }
\begin{itemize}
\item Suppose we've already seen data $\cd$.

\pause{}
\item The \textbf{posterior predictive distribution }is given by
\[
x\mapsto p(y\mid x,\cd)\pause=\int p(y\mid x;\theta)p(\theta\mid\cd)\,d\theta.
\]


\pause{}
\item This is an average of all conditional densities in our family, weighted
by the posterior. 
\pause{}
\item May not have closed form.
\pause{}
\item Numerical integral may be hard to compute.
\end{itemize}
\end{frame}
%
\begin{frame}{MAP Estimator Versus Posterior Predictive Distribution}
\begin{itemize}
\item How do we predict by posterior predictive distribution given a new data point $x^*$?
\pause{}
\item We can use $\hat{y}=\argmax_y	p(y\mid x,\cd)$
\pause{}
\item What about our MAP estimator for $\theta$?
\[
\hat{\theta} = \argmax_{\theta}p(\theta\mid\cd)
\]
\pause{}
\item We can also predict $y$ by
\[
\hat{y} = \argmax_y p(y\mid x;\theta = \hat{\theta})
\]
\item In general, the predictions from two methods are different. 
\end{itemize}	
\end{frame}
\section{Questions}
\begin{frame}{Question 1}
\textbf{Question 1.} (From DeGroot and Schervish) Let $\theta$ denote the proportion of registered voters in a large city who are in favor of a certain proposition.  Suppose that the value of $\theta$ is unknown, and two statisticians $A$ and~$B$
  assign to~$\theta$ the following different prior PDFs $\xi_A(\theta)$ and~$\xi_B(\theta)$, respectively:
  $$\begin{array}{rcll}
    \xi_A(\theta)&=&2\theta & \mbox{for $0<\theta<1$,}\\
    \xi_B(\theta)&=&4\theta^3 & \mbox{for $0<\theta<1$.}
  \end{array}$$
  In a random sample of 1000 registered voters from the city, it is found that 710 are in favor of the proposition.
  \begin{enumerate}
  \item Find the posterior distribution that each statistician assigns to~$\theta$.
  \item Find the Bayes estimate of $\theta$
    (minimizer of posterior expected loss)
    for each statistician based on the squared error loss function.
  \item Show that after the opinions of the 1000 registered voters in the random sample had been obtained, the Bayes estimates for the two statisticians could not
    possibly differ by more than $0.002$, regardless of the number in the sample who were in favor of the proposition.
  \end{enumerate}	
\end{frame}
\begin{frame}{Question 1: Solution}
Note that both prior distributions are from the Beta family.
  \begin{enumerate}
  \item We have
    $$\xi_A(\theta|x) \propto f(x|\theta)\xi_A(\theta) \propto \theta^{711}(1-\theta)^{290}$$
    and
    $$\xi_B(\theta|x)\propto f(x|\theta)\xi_B(\theta) \propto \theta^{713}(1-\theta)^{290}.$$
    Thus the posteriors from $A$ and~$B$ are both beta with parameters $(712,291)$ and $(714,291)$, respectively.
    \pause{}
  \item The respective means are $\frac{712}{1003}$ and $\frac{714}{1005}$.
    \pause{}
  \item In general the two means are given by
    $$\frac{a+2}{1003}\quad\mbox{and}\quad\frac{a+4}{1005}.$$
    The difference is less than $2/1000=.002$.
  \end{enumerate}	
\end{frame}

\begin{frame}{Question 2 and 3}
\begin{itemize}
\item \textbf{Question 2.} Two statistics students decide to compute 95\% confidence
  intervals for the distribution parameter $\theta$ using an
  i.i.d.~sample $X_1,\ldots,X_n$.  Student B uses
  Bayesian methods to find a 95\% credible set $[L_B,R_B]$ for
  $\theta$.  Student F uses frequentist methods to find a 95\%
  confidence interval $[L_F,R_F]$ for $\theta$.  Both conclude that
  parameter $\theta$ is in their respective intervals with probability at least
  $.95$.  Who is correct? Explain.
\item \textbf{Question 3.}	Suppose $\theta$ has prior distribution $\text{Beta}(a,b)$ for some $a,b>0$.  Given
  $\theta$, suppose we make independent coin flips with heads
  probability $\theta$.  Find values of $a,b$ and the coin flips so that the
  posterior variance is larger than the prior variance.  [Hint: Recall
    that a $\text{Beta}(a,b)$ random variable has variance given by
    $$\frac{ab}{(a+b)^2(a+b+1)}.$$
    Try $b=1$.]
\end{itemize}
\end{frame}

\begin{frame}{Question 2: Solution}
\begin{itemize}
\item \textbf{Question 2.} Two statistics students decide to compute 95\% confidence
  intervals for the distribution parameter $\theta$ using an
  i.i.d.~sample $X_1,\ldots,X_n$.  Student B uses
  Bayesian methods to find a 95\% credible set $[L_B,R_B]$ for
  $\theta$.  Student F uses frequentist methods to find a 95\%
  confidence interval $[L_F,R_F]$ for $\theta$.  Both conclude that
  parameter $\theta$ is in their respective intervals with probability at least
  $.95$.  Who is correct? Explain.
\item[] \textbf{Solution:} 
\begin{itemize}
\item The frequentist student is totally incorrect, since they
  have misunderstood what a frequentist confidence interval is.  Using
  frequentist methodology, $\theta$ is not a random variable, so it
  doesn't make sense to say it lies in some fixed interval
  $[L_F,R_F]$.  The correct interpretation is that if independent experiments like
  this were repeated, then at least 95\% of the time $[L_F,R_F]$ will
  contain $\theta$.  \textbf{That is, the interval is random not $\theta$.}
\item   We can say that the Bayesian student is consistent.
  Recall that to compute the credible set, the Bayesian student had to
  introduce some prior distribution $\pi$ on~$\theta$.  What we can say is
  if someone believes $\pi$ is correct, then it is rational, given the
  data, to conclude that $\theta$ will lie in the posterior credible set
  with probability 95\%.
\end{itemize}
\end{itemize}

\end{frame}

\begin{frame}{Question 3: Solution}
\begin{itemize}
\item \textbf{Question 3.}	Suppose $\theta$ has prior distribution $\text{Beta}(a,b)$ for some $a,b>0$.  Given
  $\theta$, suppose we make independent coin flips with heads
  probability $\theta$.  Find values of $a,b$ and the coin flips so that the
  posterior variance is larger than the prior variance.  [Hint: Recall
    that a $\text{Beta}(a,b)$ random variable has variance given by
    $$\frac{ab}{(a+b)^2(a+b+1)}.$$
    Try $b=1$.]
\item[] \textbf{Solution:} As hinted, let's try $a=10$, $b=1$ and 9 coin flips all
  landing tails.  The prior variance is given by
  $$\frac{10\cdot 1}{(10+1)^2(10+1+1)} = \frac{5}{726}\approx .0069$$
  while the posterior variance is given by
  $$\frac{10\cdot 10}{(10+10)^2(10+10+1)} = \frac{1}{84}\approx.0119.$$
	
\end{itemize}
	
\end{frame}

\begin{frame}{Question 4}
\textbf{Question 4.} What would be the Maximum a Posteriori (MAP) estimator for $\lambda$ for i.i.d. $\{x_1, x_2, \dots, x_N\}$ where $x_i \sim \exp(\lambda)$ with prior $\lambda \sim \text{Uniform}[u_0,u_1]$?	
\end{frame}

\begin{frame}{Question 4: Solution}	
\begin{itemize}
\item Likelihood: $L(x_1,\dots,x_N|\lambda) = \lambda e^{-\lambda(x_1+\cdots+x_N)}$
\item log-likelihood: $\ell(\lambda|x_1,\dots,x_N) = n \ln\lambda-\lambda(x_1+\cdots+x_N)$
\item $\ell'(\lambda) = \frac{n}{\lambda}-(x_1+\dots+x_N)\begin{cases}
 	>0 \quad \text{if}~0<\lambda<\bar{x}=(x_1+\cdots+x_N/N),\\
 	=0 \quad \text{if}~\lambda=\bar{x}\\
 	<0 \quad \text{if}~\lambda>\bar{x}
 \end{cases}
$
\item Prior: $p(\lambda) = \frac{1}{u_1-u_0} \mathds{1}_{[u_0,u_1]}(\lambda)$.
\item Posterior: $p(\lambda|x_1,\dots,x_N)\propto L(x_1,\dots,x_N|\lambda)p(\lambda) = \lambda e^{-\lambda(x_1+\cdots+x_N)} \mathds{1}_{[u_0,u_1]}(\lambda)$
\item Maximum value of posterior is attained at 
\[
\lambda = \begin{cases}
 	u_0 \quad \text{if}~u_0>\bar{x},\\
 	\bar{x} \quad \text{if}~u_0\leq\bar{x}\leq u_1\\
 	u_1 \quad \text{if}~u_1<\bar{x}.
 \end{cases}
 \]
\end{itemize}
\end{frame}

\end{document}
